\chapter{Our Dynamic Universe}

\section{Equations of Motion}
\begin{multicols}{2}
	\begin{itemize}
        \item Can I calculate the equivalent vector by scale diagram or
            otherwise for vectors in a non-right angled triangle?

         \item Can I carry out calculations to find the horizontal and
            vertical components of vectors using the relationships: \\
            $ V_{horizontal} = V cos(\theta) $ \qquad
            $ V_{vertical} = V sin(\theta) $

        \item Can I carry out calculations using the kinematic relationships
            $ v = u + at $ \qquad
            $ s = ut * \frac{1}{2}at^2 $ \qquad
            $ v^2 = u^2 - 2as $ \qquad
            $ s = (u + v) t $

            for objects moving with a constant acceleration in a straight line?

        \item Can I interpret displacement-time graphs? e.g. gradient is
           velocity

        \item Can I interpret velocity-time graphs including:
            \begin{enumerate}
               \item area under graph is displacement?
               \item gradient is acceleration?
               \item objects in freefall taking into account air resistance
                   and changing surface area?
            \end{enumerate}

        \item Can I draw and interpret acceleration-time graphs using
              information obtained from a velocity-time graph for motion with
              constant acceleration?

        \item Can I identify and interpret motion-time graphs of:
             \begin{enumerate}
                \item bouncing objects?
                \item objects thrown vertically upwards?
             \end{enumerate}

        \end{itemize}
    \end{multicols}

    \section{Forces, Energy and Power}
    \begin{multicols}{2}
        \begin{itemize}

        \item Can I analyse the motion of an object using free body
            diagrams and Newton’s first and second laws?

        \item Can I carry out calculations using Newton’s second law
            $(F=ma)$ in one direction only?

        \item How does the direction of frictional forces compare to the
            direction of motion of an object?

        \item Can I carry out calculations using Newton’s second Law
            $(F=ma)$ when a number of opposing forces act on an object
            in the horizontal direction?
            
        \item Can I analyse and carry out calculations using Newton’s
            second law $(F=ma)$ and $W=mg$ when a number of opposing
            forces act on an object in the vertical direction? e.g rockets,
            lifts etc

        \item Can I carry out calculations using Newton’s second law
            (F=ma) when investigating internal forces (Tension exerted
            by a string or cable) e.g. car pulling a caravan etc?

        \item Can I carry out calculations using Newton’s second law
            (F=ma) when an object is on an incline (slope)?
            (components of $weight = mg\sin\theta$ \& $mg\cos\theta$)

        \item Can I analyse and calculate the horizontal and vertical
            component of vectors (including forces)?

        \item Can I carry out energy calculations involving work
            done, potential energy, kinetic energy and power in
            \begin{enumerate}
                \item familiar situations
                \item unfamiliar situations?
            \end{enumerate}

        \item Can I carry out calculations and analyse situations
            involving the conservation of energy?

        \end{itemize}
    \end{multicols}

    \section{Collisions, Explosions and Impulse}
    \begin{multicols}{2}
        \begin{itemize}
            \item Can I carry out calculations using the equation p=mv?
            \item Can I state the law of conservation of momentum?
            \item Can I carry out calculations using p before = p after for
                collisions between objects moving in the same direction?
            \item Can I carry out calculations using p before = p after for
                collisions between objects moving in opposite directions?
            \item What is meant by an
                \begin{enumerate}
                    \item elastic collision
                    \item inelastic collision
                \end{enumerate}
            \item Can I use the equation $E = \frac{1}{2}mv^2$ to establish
                whether a collision is elastic or inelastic?
            \item Can I carry out calculations using $p before = p after$ for
                explosions in one dimension?
            \item Can I apply the law of conservation of momentum to the
                interaction of two objects moving in one dimension to
                show that the forces acting on each object are equal in size
                and opposite in direction.
            \item Can I carry out calculations using the equation $Impulse =
                Force \times time\, of\, contact$?
            \item How does Impulse and change in momentum compare in
                size during a collision in one dimension?
            \item Can I carry our calculations using the equation 
                $Ft = mv - mu$?
            \item Can I identify the shape of a force-time graph of a collision
                in one dimension?
            \item Can I interpret force-time graphs including:
                \begin{enumerate}
                    \item area under graph is impulse
                    \item changing time of impact to see the effect of the
                        average force and impulse e.g. use of crumple zones
                        and air bags
                \end{enumerate}
        \end{itemize}
    \end{multicols}

    \section{Gravitation}
    \begin{multicols}{2}
        \begin{itemize}
            \item How does the vertical motion of a dropped object compare
                with an object which has been projected horizontally?
                
            \item Can I describe the vertical motion of an object which has
                been projected
            \begin{enumerate}
                \item horizontally
                \item upwards at an angle (oblique)?
            \end{enumerate}

            \item Can I describe the horizontal motion of an object which has
                been projected
            \begin{enumerate}
                \item horizontally
                \item upwards at an angle?
            \end{enumerate}

            \item Can I carry out calculations using
                \begin{enumerate}
                    \item $d = vt$ (horizontal component)\\
                    \item $v = u + at$, $s = ut + \frac{1}{2}at^2$, 
                        $v^2 = u^2 + 2as$ (vertical component)
                \end{enumerate}
                for objects projected horizontally?

            \item Can I carry out calculations using
                \begin{enumerate}
                    \item $d = vt$ (horizontal component)\\
                    \item $v = u + at$, $s = ut + \frac{1}{2}at^2$, 
                        $v^2 = u^2 + 2as$ (vertical component)
                \end{enumerate}
                for objects projected at an angle?

            \item Using Newton’s thought experiment, can I explain how
                satellites remain in orbit?

            \item What does the magnitude of the gravitational field depend
                upon?

            \item How do scientists believe stars were formed?

            \item Can I carry out calculations using the equation \\
                \[ F = G\frac{m_1 m_2}{r^2} \]

            \item Can I state an application of gravitational force?
        \end{itemize}
    \end{multicols}

    \section{Special Relativity}
    \begin{multicols}{2}
        \begin{itemize}
            \item Do I know that the speed of light in a vacuum is the same
                for all observers in all reference frames

            \item Can I describe the motion of an object in terms of an
                observer’s frame of reference, using time dilation and
                length contraction?

            \item Can I carry out calculations involving time dilation?

            \item Can I carry out calculations involving length contraction?

            \item Do I know the minimum speed at which relativistic effects
                are observed
        \end{itemize}
\end{multicols}

\section{The Expanding Universe}
\begin{multicols}{2}
	\begin{itemize}
        \item Can I explain what is meant by the Doppler effect?

        \item Can I state which types of waves undergo the Doppler
            effect?

        \item Can I calculate the apparent frequency detected by a
            stationary observer relative to a moving source of sound
            waves?, i.e.
        \item Can I explain what is meant by redshift?
        \item Can I carry out calculations using

            to calculate the redshift of a galaxy?
        \item Can I carry out calculations using

            to calculate the redshift of a galaxy at non-relativistic speeds?

        \item Can I carry out calculations using Hubbles' law, i.e.
            \begin{itemize}
                \item Can I explain how hubble's law allows use to estimate the
                    age of the universe?
                \item Can I describe the evidence which has lead to the theory
                    that the universe is expanding?
                \item Can I explain how the rate of expansion of the universe
                    is changing and name the force responsible for this?
            \end{itemize}

        \item Can I describe how observations can be used to estimate the mass
            of our galaxy?

        \item Do I know what is meant by the term dark matter?

        \item Can I describe the evidence for dark matteR?

        \item Do I know what is meant by the term dark energy?

        \item Can I describe the evidence for dark energy?
	\end{itemize}
\end{multicols}

\section{The Big Bang}
\begin{multicols}{2}
	\begin{itemize}
        \item Can I describe the relationship between the temperature of a
            stellar object and the wavelength distribution of radiation
            it emits?

        \item Do I know how the peak wavelength of emitted radiation is
            related to the object's wavelength?

        \item Do I know how the intensity of radiation is related to the
            temperature of the star?

        \item Do I know what is meant by the cosmic microwave background
            radiation?

        \item Can I describe evidence to justify the Big Bang as a theory for
            the beginning and evolution of the universe?
	\end{itemize}
\end{multicols}
