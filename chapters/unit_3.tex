\chapter{Electricity}

\section{Monitoring and Measuring A.C.}
\begin{multicols}{2}
	\begin{itemize}
        \item Describe a.c. electric current and voltage in terms of the movement of
            charges in a circuit.
        \item State that a.c. current and voltage can be measured using an
            oscilloscope.
        \item Describe how to measure the frequency and peak voltage of an
            alternating supply using an oscilloscope.
        \item State that the r.m.s. voltage is equivalent to a d.c. voltage that
            produces the same power.
        \item State the relationship between peak and r.m.s. values for a
            sinusoidally varying voltage and current.
        \item Carry out calculations involving peak and r.m.s. values of voltage and
current.
	\end{itemize}
\end{multicols}

\section{Current, Voltage, Power and Resistance}
\begin{multicols}{2}
	\begin{itemize}
        \item State that voltage is defined as the energy transformed per unit of
            charge.
        \item State the relationship $V = \frac{E_w}{Q}$.
        \item Carry out calculations involving the relationship between energy,
            voltage and charge
        \item State that the energy transformed from an external source to the
            circuit is known as the electromotive force (e.m.f.).
        \item Give examples of sources of e.m.f.
        \item State that the energy transformed into another form of energy by a
            circuit component is known as the potential difference (p.d.)
        \item Carry out calculations involving the relationships between resistances
            in potential dividers using the potential divider equation and Ohm’s
            law.
	\end{itemize}
\end{multicols}

\section{Electrical Sources and Internal Resistance}
\begin{multicols}{2}
	\begin{itemize}
        \item State that a power supply is equivalent to a source of e.m.f. with a
            resistor in series, the internal resistance.
        \item Describe the principles of a method for measuring the e.m.f. and
            internal resistance of a source
        \item Explain why the e.m.f. of a source is equal to the open circuit p.d.
            across the terminals of a source.
        \item State that the closed circuit p.d. across the terminals of a source is
            equal to the t.p.d.
        \item State that the e.m.f. of a cell is equal to the sum of the t.p.d. and the
            lost volts.
        \item Carry out calculations involving the relationship between the e.m.f.,
            t.p.d. and lost volts.
        \item Describe two methods of measuring e.m.f. and internal resistance by
            graphical methods.
        \item State the R = r for maximum transfer of energy between a source and
            a load.
	\end{itemize}
\end{multicols}

\section{Capacitors}
\begin{multicols}{2}
	\begin{itemize}
        \item State that the capacitance of a capacitor is a measure of its ability to
            store charge.
        \item State that a simple capacitor consists of two parallel conducting
            plates separated by an air gap.
        \item Describe the circuit symbol for a capacitor.
        \item State that the charge Q stored on a capacitor is directly proportional
            to the p.d. V across it.
        \item Describe the principles of a method to show that the p.d. across a
            capacitor is directly proportional to the charge on the plates.
        \item State that capacitance is defined as the gradient of the charge against
            p.d. graph or the ratio of charge to p.d.
        \item State that the unit of capacitance is the farad and that one farad is one
            coulomb per volt.
        \item Carry out calculations involving the relationship between charge,
            capacitance and p.d.
        \item Explain why work must be done to charge a capacitor.
        \item State that the work done to charge a capacitor is given by the area
            under the graph of charge against p.d.
        \item State that the energy stored in a capacitor is given by 1⁄2 (charge ×
            p.d.) and equivalent expressions
        \item Carry out calculations using the relationship between energy, charge
            and p.d. or alternative expressions
        \item Draw qualitative graphs of current against time and of voltage against
            time for the charge and discharge of a capacitor in a d.c. circuit
            containing a
        \item resistor and capacitor in series.
        \item Carry out calculations involving voltage and current in CR circuits.
	\end{itemize}
\end{multicols}

\section{Conductors, Semiconductors and Insulators}
\begin{multicols}{2}
	\begin{itemize}
        \item State that solids can be classified into three types according to their
            electrical properties as conductors, semiconductors and insulators.
        \item Give examples of conductors, semiconductors and insulators.
        \item State that the different electrical properties of conductors,
            semiconductors and insulators can be explained by Band Theory.
        \item State that in isolated atoms, the permitted energy levels consist of a
            series of sharply defined states.
        \item State that in solids, the permitted energy levels associated with each
            state of the isolated atom forms a continuous band.
        \item State that the two highest bands are known as the valence band and
            the conduction band, respectively.
        \item State that the valence band contains electrons that can be considered
            to be bound to the atom.
        \item State that the valence band is full in insulators and semiconductors.
        \item State that the conduction band contains electrons that are free to
            move.
        \item State that the conduction band is empty in insulators and
            semiconductors, but partially filled in conductors.
        \item State that only partially filled bands may permit conduction.
        \item State that there is an energy gap between the valence and conduction
            bands in insulators and semiconductors.
        \item State that an electron can absorb energy to move between the valence
            band and the conduction band.
        \item State that in insulators, the energy gap is normally too large for
electrons to jump to the conduction band.
• State that in semiconductors, the energy gap is much smaller and
electrons can jump to the conduction band as a result of thermal
excitation.
        \end{itemize}
\end{multicols}

\section{Intrinsic and Extrinsic Semiconductors}
\begin{multicols}{2}
	\begin{itemize}
        \item State that in semiconductors, conduction occurs by means of 
            negative charge carriers, (electrons) or positive charge carriers 
            (holes).
        \item State that in pure semiconductors there are very few electrons
            available to coduct which makes the resistance very large
        \item State that in pure semiconductors more free electrons become
            available at higher temperatures, therefore the conductivity
            increases and the resistance decreases
        \item State that these pure semiconductors are known as intrinsic
            semiconductors
        \item State that the additin of impurity atoms to a pure semiconductor
            (a process called doping) increases its conductivity by either
            adding electrons or holes to the lattice
        \item State that doped semiconductors now have a majority charge
            carrier present and are known as extrinsic semiconductors
        \item State that group V doping agents result in N-type extrinsic
            semiconductors, which contain extra electrons
        \item State that group III dopiing agents result in P-type extrinsic
            semiconductors, which contain extra holes.
        \item Explain how doping can form an n-type semiconductor in which the
            majority of the charge carriers are negative, or a p-type
            semiconductor in which the majority of charge carriers are
            negative, or a p-type semiconductor in which the majority of
            the charge carriers are positive.
	\end{itemize}
\end{multicols}

\section{P - N Junctions}
\begin{multicols}{2}
	\begin{itemize}
        \item State that the interface between a p-type and n-type 
            semiconductor is called a p-n junction and it functions as a diode
        \item State that the majority charge carriers diffuse towards the
            junction and electrons and holes combine to form ions.
        \item State that this results in a depletion zone accross the p-n
            junction where the density of charge carriers is low, with
            positive ions on the n-type side and negative ions on the p-type
            side
        \item State that when the p-type material is connected to the positive
            terminal of a supply and the n-type to the negative terminal, then
            the junction is forward biased.
        \item State that if the potential difference accross the junction is
            sufficient to force electrons to cross the depletion zone, then the
            junction will conduct.
        \item State that when the terminals are reversed, the junction is
            reverse biased and cannot conduct
        \item Describe the movement of the charge carrers in a forward/reverse
            biased p-n junction diode
        \item State that in a light emitting diode a large forward bias is
            applied to the p-n junction enabling positive and negative charge
            carriers to recombine, producing photons of light.
        \item State that the frequency of the emitted photons increases as the
            size of the energy gap between the conduction and valence bands
            increases
        \item State the relationship $E = h f$. 
        \item Carry out calculations involving the relationships between
            $E$, $h$, $f$ and $\lambda$.
        \item State that in photovoltaic cells, absorbed photons can create
            electron-hole pairs to produce a potential difference.

	\end{itemize}
\end{multicols}
