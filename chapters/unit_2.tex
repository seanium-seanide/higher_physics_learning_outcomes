\chapter{Particles and Waves}

\section{Orders of Magnitude}
%\begin{multicols}{2}
	\begin{itemize}
        \item To be able to discuss the range of orders of magnitude of length 
            from the very small (sub-nuclear) to the very large (distance to 
            furthest known celestial objects).
	\end{itemize}
%\end{multicols}

\section{The Standard Model}
\begin{multicols}{2}
	\begin{itemize}
        \item To be able to discuss the evidence for the sub-nuclear particles 
            and the existence of antimatter.

        \item State that Fermions, the matter particles, consist of Quarks 
            (6 types) and the 6 Leptons (Electron, Muon and Tau, together with 
            their neutrinos).

        \item State that Hadrons are composite particles made of Quarks

        \item State that Baryons are made of three Quarks and Mesons are made 
             of two Quarks.

        \item To be able to describe beta decay as the first evidence for the 
             neutrino
        \item State that the force mediating particles are bosons (Photons, W 
            and Z Bosons and Gluons)
        \item Describe how a PET scanner works
	\end{itemize}
\end{multicols}

\section{Electric Fields}
\begin{multicols}{2}
	\begin{itemize}
        \item State that, in an electric field, an electric charge experiences
            a force
        \item State that an electric field applied to a conductor causes the 
            free electric charges in it to move
	\end{itemize}
\end{multicols}

\section{Movement of Charge}
\begin{multicols}{2}
	\begin{itemize}
        \item State that work, W, is done when a charge, Q, is moved in an 
            electric field.

        \item State that the potential difference (V) between two points is a
            measure of the work done in moving one coulomb of charge between
            the two points

        \item State that if one joule of work is done moving one coulomb of 
            charge between two points, the potential difference between the 
            points is one volt.

        \item State the relationship $V = \frac{W}{Q}$.

        \item Carry out calculations involving the relationship
            $V = \frac{W}{Q}$

        \item Calculate the speed of a charged particle accelerated in an 
            electric field using the relationship 
            $QV = \frac{1}{2} mv^2$ .
	\end{itemize}
\end{multicols}

\section{Charged Particles in a Magnetic Field}
\begin{multicols}{2}
	\begin{itemize}
        \item State that a moving charge produces a magnetic field
        \item Describe the force acting on a charged particle in a magnetic 
            field.
	\end{itemize}
\end{multicols}

\section{Particle Accelerators}
\begin{multicols}{2}
	\begin{itemize}
        \item State the three types of particle accelerator
        \item Describe the basic operation of particle accelerators in terms of
            acceleration, deflection and collision of charged particles
	\end{itemize}
\end{multicols}

\section{Fission and Fusion}
\begin{multicols}{2}
	\begin{itemize}
        \item Explain what is meant by alpha, beta and gamma decay of
            radionuclides

        \item State that in fission a nucleus of large mass number splits into 
            two nuclei of smaller mass numbers, usually along with several 
            neutrons

        \item State that fission may be spontaneous or induced by neutron
            bombardment

        \item State that in fusion two nuclei combine to form a nucleus of 
            larger mass number

        \item Explain, using E = mc2, how the products of fission and fusion
            acquire large amounts of kinetic energy

        \item Identify the processes occurring in nuclear reactions written in
            symbolic form

        \item Carry out calculations using E = mc2 for fission and fusion 
            reactions.

        \item Describe the principles of the operation of a nuclear fission 
            reactor in terms of fuel rods, moderator, control rods, coolant and
            containment vessel

        \item Describe the coolant and containment issues in nuclear fusion
            reactors.
	\end{itemize}
\end{multicols}

\section{The Photoelectric Effect and Wave Particle Duality}
\begin{multicols}{2}
	\begin{itemize}
        \item State that photoelectric emission from a surface occurs only if the
            frequency of the incident radiation is greater than some threshold
            fo, which depends on the nature of the surface 

        \item State that a beam of radiation can be regarded as a stream of
            individual energy bundles called photons, each having an energy
            E = hf, where h is Planck’s constant and f is the frequency of the
            radiation.

        \item Carry out calculations involving the relationship E = hf


        \item State that photoelectrons are ejected with a maximum kinetic energy,
             Ek, which is given by the difference between the energy of the
            incident photon hf and the work function hfo of the surface:
            Ek = hf – hfo.

        \item State that for frequencies smaller than the threshold value, an increase
            in the irradiance of the radiation at the surface will not cause
            photoelectric emission

        \item State that for frequencies greater than the threshold value, the
            photoelectric current produced by monochromatic radiation is
            directly proportional to the irradiance of the radiation at the surface.

        \item Explain that if N photons per second are incident per unit area on a
            surface, the irradiance at the surface I = Nhf.
	\end{itemize}
\end{multicols}

\section{Conditions for Constructive and Destructive Interference}
\begin{multicols}{2}
	\begin{itemize}
        \item  Use correctly in context the terms: ‘in phase’, ‘out of phase’ and
            ‘coherent’, when applied to waves
        \item Explain the meaning of: ‘constructive interference’ and ‘destructive
            interference’ in terms of superposition of waves.
        \item tate that interference is the test for a wave
	\end{itemize}
\end{multicols}

\section{Interference of Waves Using Two Coherent Sources}
\begin{multicols}{2}
	\begin{itemize}
        \item State the conditions for maxima and minima in an interference
            pattern formed by two coherent sources in the form:
        \item Path difference = nλ for maxima and Path difference = (n + 1⁄2) λ for 
            minima, where n is an integer
        \item Carry out calculations using the above relationships
	\end{itemize}
\end{multicols}

\section{Gratings}
\begin{multicols}{2}
	\begin{itemize}
        \item Describe the effect of a grating on a monochromatic light beam
        \item Carry out calculations using the grating equation ;
            dsinθ = nλ
        \item Describe the principles of a method for measuring the wavelength of
            a monochromatic light source, using a grating
        \item State approximate values for the wavelengths of red, green and blue
            light.
        \item Describe and compare the white light spectra produced by a grating
            and a prism
	\end{itemize}
\end{multicols}

\section{Refraction}
\begin{multicols}{2}
	\begin{itemize}
        \item State that the ratio $\frac{\sin\theta_1}{\sin\theta_2}$
            is a constant when light passes obliquely from medium 1 to medium
            2
        \item State the absolute refractive index, n, of a medium is the ratio
        \item sinθ1 /sinθ2, where θ1 is in a vacuum (or air as an approximation)
            and θ2 is in the medium
        \item Describe the principles of a method for measuring the absolute
            refractive index of glass for monochromatic light
        \item Carry out calculations using the relationship for refractive index
        \item State that the refractive index depends on the frequency of the
            incident light.
        \item State that the frequency of the wave is unaltered by a change in
            medium
        \item State that the relationship for refraction of a wave from medium
            1 to medium 2 is

            \[ \frac{\sin\theta_1}{\sin\theta_2} = \frac{\lambda_1}{\lambda_2} = \frac{v_1}{v_2}\]
        \item Carry out calculations using the above relationships
	\end{itemize}
\end{multicols}

\section{Critical Angle and Total Internal Reflection}
\begin{multicols}{2}
	\begin{itemize}
        \item Explain what is meant by total internal reflection
        \item Explain what is meant by critical angle θc
        \item Describe the principles of a method for measuring a critical angle
        \item Derive the relationship sinθc = 1/n, where θc is the critical angle for a
            medium of absolute refractive index, n.
        \item Carry out calculations using the above relationship
	\end{itemize}
\end{multicols}


\section{Irradiance and the Inverse Square Law}
\begin{multicols}{2}
	\begin{itemize}
        \item State that the irradiance I at a surface on which radiation is incident is
            the power per unit area
        \item Describe the principles of a method for showing that the irradiance is
            inversely proportional to the square of the distance from a point source
        \item Carry out calculations involving the relationship $I = \frac{k}/{d2}$
        \item Explain why a beam of laser light having a power even as low as 0.1
            mW may cause eye damage
	\end{itemize}
\end{multicols}

\section{Spectra}
\begin{multicols}{2}
	\begin{itemize}
        \item State that electrons in a free atom occupy discrete energy levels
        \item Draw a diagram which represents qualitatively the energy levels of a
            hydrogen atom
        \item  Use the following terms correctly in context: ground state, excited
            state, ionisation level
        \item State that an emission line in a spectrum occurs when an electron
            makes a transition between an excited energy level W2 and a lower
            level W1, where W2 – W1 = hf
        \item State that an absorption line in a spectrum occurs when an electron in
            energy level W1 absorbs radiation of energy hf and is excited to
            energy level W2, where W2 = W1 + hf.
        \item Explain the occurrence of absorption lines in the spectrum of
            sunlight.
	\end{itemize}
\end{multicols}
